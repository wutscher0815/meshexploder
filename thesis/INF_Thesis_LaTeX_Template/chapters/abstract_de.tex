%
% Vienna University of Technology - Faculty of Informatics
% TUINF thesis - Abstract German
%
% For questions and comments send an email to
% Thomas Auzinger <thomas.auzinger@cg.tuwien.ac.at>
% Thomas Krennwallner <tkren@kr.tuwien.ac.at>
%
% ---------------------------------------------------------
%
% TU Wien - Fakultät für Informatik
% TUINF Arbeit - Kurzfassung deutsch
%
% Für Fragen oder Kommentare schicken Sie eine Email an
% Thomas Auzinger <thomas.auzinger@cg.tuwien.ac.at>
% Thomas Krennwallner <tkren@kr.tuwien.ac.at>
%

\selectlanguage{ngerman}

\chapter*{Kurzfassung}

Die Explosionsgrafik ist eine Technik der Illustrativen Visualisierung, die die Funktion oder den Aufbau eines komplexen Objekts darstellt, in dem es in seine Einzelteile zerlegt wird, die dann so im Raum platziert werden, dass sie im Idealfall vollständig sichtbar sind und sich durch ihre Position auf ihre ursprüngliche Position innerhalb des Objekts rückschließen lässt. Diese Bachelorarbeit beschäftigt sich mit der Implementierung eines Plug-Ins für die Visualisierungssoftware VolumeShop, das in der Lage ist aus zusammengesetzten Dreiecks-Meshes einfache Explosionsgrafiken dynamisch zu erzeugen. Um aussageschwache Resultate bei Betrachtung von bestimmten Blickwinkeln aus zu vermeiden habe ich für diese Fälle Explosionszeichnung mit Ghosting kombiniert. Ghosting ist eine andere Illustrationstechnik, bei der Objekte, die sich zwischen dem Betrachter und einem wichtigen Objekt befinden transparent dargestellt werden.