\chapter{State of the Art Description}
\section{Exploded Views}
\subsection{What are the main problems when creating exploded views?}

To create an exploded view,  a few considerations have to be made Li et al. provide a comprehensive list of these challenges in their 2008 paper\cite{proc:Li:2008:AGI}.\\
First of all what do I want to tell with the image, in essence what are the objects of interest that should attract the viewers attention.\\
Also one has to consider in which direction the parts should be displaced. 
\begin{quote}
Many objects have a canonical coordinate frame that may be defined by a number of factors, including symmetry , real-world orientation, and domain-specific conventions. In most exploded views, parts are exploded only along these canonical axes.\cite{proc:Li:2008:AGI}
\end{quote}
Especially if the graphic conveys technical instructions or how-things-work-descriptions descriptions these axes are usually the directions in which they are assembled \cite{Agra03} or - if applicable- the axis around which they rotate\cite{MitraYYLA13}.
If the interesting parts are inside a container that is also part of the object, this container is split and its segments are also exploded. In the most simple approach, which is what I implemented, the cutting plane is the normal plane of the explosion direction, with the center of the object's bounding box.\\
In more complex objects hierarchy of the exploding parts is another challenge to face. 
\begin{quote}
In many complex models, individual parts are grouped into sub-assemblies (i.e., collections of parts).\cite{proc:Li:2008:AGI}
\end{quote}
Take for example an object that has a container and a lot of small parts that inside whose function should be clarified by exploding them in their canonical direction.  In this case it might be convenient to split and explode the container along a different axis than the internal parts to get a more compact visualization.\\
Another challenge is the decision how far to displace the objects. Ideally each part should be fully visible but if there are many objects to be exploded or system the displacement direction is similar to the viewing direction, the size of the graph would grow to be enormous therefore resulting in loss of detail and expressiveness of the visualization. That may make it necessary to have some overlap as a trade-off to retain the compactness of the visualization.\\
With a freely rotatable and movable viewpoint the user can avoid visual clutter or lack of compactness(depending on how the system behaves) by choosing a viewpoint that provides as little clutter as possible,which narrows down the possibilities of expressive viewpoints to a minimum. A solution would be to use a view dependent force-based displacement behavior as suggested by Bruckner and Gr\"oller\cite{proc:bruckner-2006-EVV}: Each exploding part is being displaced by a sum of multiple forces:
\begin{itemize}
\item \textbf{Explosion force} This force is pushing the part away from its original location, its magnitude is indirectly proportional to $e^{||r||}$ where $r$ is the distance between the object and the explosion point
\item \textbf{Spacing force} This repulsive force that each exploded part effects on all other exploded parts prevents parts from clustering and is indirectly proportional to $r^2$ where $r$ is the distance between the two parts.
\item \textbf{Viewing force}Additionally,  a viewing force is introduced that pushes the parts away from the viewing ray, thus preventing occlusions. It is indirectly proportional to the distance $r$ between the viewing ray and the part.
\end{itemize}


\subsection{What solutions for these Problems exist?}
Explosion graph - Hierarchical explosion graph - force-based placement
\subsection{What are the advantages and disadvantages of exploded views?}

\section{Ghosting}
\subsection{What is the purpose of Ghosting?}
show interior or obstructed parts - information about foreground hinted - 
\subsection{What are the main problems when visualizing objects using Ghosting?}
good values for $\alpha$ - smart visibility - 

\subsection{What are the advantages and disadvantages of Ghosting?}
Objects can be visualized ``Where they Are'' - compact - some information always lost -
\section{Combination of these two techniques}
