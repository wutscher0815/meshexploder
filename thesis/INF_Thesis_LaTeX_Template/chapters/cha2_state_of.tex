\chapter{State of the Art Description}
\section{Exploded Views}
\subsection{What is the purpose of Exploded Views?}

\begin{quote}
in an exploded view the object is decomposed into several parts which are displaced so that internal details are visible \cite{BruGr2006}
\end{quote}
Composite objects - View that makes each component fully visible - composition is implied by the positions of the objects - Examples: building instructions, Explanation of complex technical objects
\subsection{What are the main problems when creating exploded views?}
Which direction - How far apart - How much overlap - Which part shall be exploded - Cutaway - Hierarchy
\subsection{What solutions for these Problems exist?}
Explosion graph - Hierarchical explosion graph
\subsection{What are the advantages and disadvantages of exploded views?}

\subsection{What were the implications of this for the project?}

\section{Ghosting}
\subsection{What is the purpose of Ghosting?}
show interior or obstructed parts - information about foreground hinted - 
\subsection{What are the main problems when visualizing objects using Ghosting?}
good values for \alpha - smart visibility - 

\subsection{What are the advantages and disadvantages of Ghosting?}
Objects can be visualized "Where they Are" - compact - some information always lost -
\section{Conclusion - Combination of these two techniques}
Problem: automatic distance for Cutaways with fixed plane - some parts leave the screen -> maximum distance, after that ghosting

