\chapter{State of the Art Description}
\section{Exploded Views}
\subsection{What is the purpose of Exploded Views?}
Exploded views are a technique used in illustrative visualization, where complex real world objects are drawn in a way that conveys how these objects are built, how they are assembled or how they work. Typically these objects are systems of interacting parts that may occlude each other or even be completely hidden by an outer hull of the visualized objects.
Examples for this would be a motor that consists of many independent parts, where the Illustration should give an idea of how it might work, a chest of drawers bought as a set of boards and connectors, that has a visual assembly instruction or the depiction of an anatomic feature.
Several techniques to tackle this problem exist, I chose to investigate further upon exploded views:
\begin{quote}
In an exploded view the object is decomposed into several parts which are displaced so that internal details are visible \cite{BruGr2006}
\end{quote}
So for technical illustrations the parts that constitute a machine could be moved on an axis so that they are all visible and their position relative to this axis would still give information as to where the place of each part would be inside the machine. As for the assembly instruction the displacement would happen along the axes they will be put together to make it clear to the viewer how to assemble the object.
The goal of my work was to create an interactive visualization that would dynamically generate exploded views.
\subsection{What are the main problems when creating exploded views?}
To create an exploded view,  a few considerations have to be made: first of all what do I want to tell with the image, in essence what are the interesting parts of the object that should attract the viewers Attention.
Also one has to consider in which direction the parts should be displaced. This can be achieved automatically by principal component analysis.
If there is a hull to the object, it is split in two parts. In the most simple approach, which is what implemented the cutting plane is the normal plane of the explosion direction, with the centre of the object's bounding box. With more complex objects and multiple objects exploded this direction might vary.
Another challenge is the decision how far to displace the objects. Ideally each part should be fully visible but if there are many objects to be exploded or system the displacement direction is similar to the viewing direction, the size of the graph would grow to be enormous therefore resulting in loss of detail and expressiveness of the visualization. That makes it necessary to have some overlap.
In more complex cases a problem is hierarchy
Which direction - How far apart - How much overlap - Which part shall be exploded - Cutaway - Hierarchy
\subsection{What solutions for these Problems exist?}
Explosion graph - Hierarchical explosion graph - force-based placement
\subsection{What are the advantages and disadvantages of exploded views?}

\subsection{What were the implications of this for the project?}

\section{Ghosting}
\subsection{What is the purpose of Ghosting?}
show interior or obstructed parts - information about foreground hinted - 
\subsection{What are the main problems when visualizing objects using Ghosting?}
good values for \alpha - smart visibility - 

\subsection{What are the advantages and disadvantages of Ghosting?}
Objects can be visualized "Where they Are" - compact - some information always lost -
\section{Conclusion - Combination of these two techniques}
Problem: automatic distance for Cutaways with fixed plane - some parts leave the screen -> maximum distance, after that ghosting
